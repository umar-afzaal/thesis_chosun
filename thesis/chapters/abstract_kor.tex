\begin{center}
\section*{한 글 요 약}
\end{center}
\addcontentsline{toc}{section}{한 글 요 약}

\bigskip
\bigskip
\centerline{\large{진화기반 신뢰성 높은 회로 합성 연구}}
\bigskip

\begin{flushright}
아프잘 우마\\
지도 교수: 이정아\\
컴퓨터공학과\\
대학원, 조선대학교\\
\end{flushright}

신뢰성 높은 회로는, 오류가 발생한 경우에도 회로의 올바른 기능을 유지하여 시스템이 정상적인 작동 상태를 유지하거나 최소한의 단계별 오류 조치로 작동할 수 있게 한다. 모든 회로에는 회로 고유의 오류 허용 내결함성이 있다. 그러나 이러한 오류에 대한 회로 고유의 복원력은 주어진 응용분야에 대부분 충분하지 않다. 이는 기존 회로 합성 도구가 면적, 전력소모, 실행시간을 고려한 회로 최적화에 초점을 일반적으로 맞추고 있기 때문이다. 오류에 의한 회로 동작이 치명적인 결과로 이어질 수 있는 응용 분야는 다양하다. 이러한 응용분야에서는, 오류 모사 입력에 의한 분석을 통하여, 하나의 오류인 경우에도 회로 작동에 매우 심각한 영향을 끼칠 수 있음을 보일 수 있다.

본 논문에서, 신뢰성 높은 회로를 자동적으로 합성하는 방법론으로, 소규모 단위의 로직 변환을 고려하는 진화 기반의 SYFR (SYnthesis of Fault-Resilient circuits) 방법론을 제안한다. 본 방법을 이용하여 신뢰성 높은 회로를 직접적으로 합성한 결과는 60개까지의 로직 입력을 가질 수 있음을 보였다. SYFR 기법은, 분할된 회로에 반복적으로 적용하는 방식을 통하여, 대형 회로에 적용할 수 있는데, 회로의 신뢰도와 회로 구현 비용 조율을 통하여 다양한 설계공간 탐색이 가능하다. 그리고, SYFR 방법론은 오류 발생시 고장 복원력이 있는, 신뢰성 높은 설계 회로 합성에 워크로드 기반으로 선택적으로 유연하게 적용할 수 있다.  본 논문에서 신뢰성 높은 회로의 설계 공간을 줄이기 위한 모집단 씨앗 메커니즘을 새롭게 제안하였으며, 실험결과로 이의 효용성을 보였다.  본 논문은 SYFR 방법론이 신뢰성 높은 회로를 합성하기 위한 경쟁력 있는 방법론으로 간주될 수 있음을 보여준다.


