\begin{center}
\section*{ABSTRACT}
\end{center}
\addcontentsline{toc}{section}{ABSTRACT}

\bigskip
\bigskip
\centerline{\large{Evolutionary Synthesis of Reliable Digital Circuits}}
\bigskip

\begin{flushright}
Umar Afzaal\\
Advisor: Prof. Lee, Jeong A\\
Department of Computer Engineering\\
Graduate School of Chosun University\\
\end{flushright}

In the event of an upset, fault-resilient circuits
maintain correct functionality allowing the system to remain fully
operational or at least operate with a graceful degradation. Every
circuit has a certain level of inherent resilience to faults. Often
times, this inherent resilience to faults is insufficient for the given
application. This is because conventional synthesis tools generally
only focus on optimizing a circuit with respect to area, power or
timing budgets. There is a wide range of applications where faulty
circuit behavior can lead to fatal results. Fault injection analyses
are reported and show that even a single fault can be critical to
the desired circuit operation in such applications.

To which end, in this thesis, we present SYFR, an evolutionary method
for automated synthesis of increased fault-resilience digital circuits
suitable for fine-grained use. Tests results for synthesis of up to 60
input circuits with SYFR are reported. SYFR can be repeatedly applied
to a circuit to obtain various design tradeoffs between fault-
resilience and implementation costs. SYFR can also be flexibly
applied to build circuits which are selectively fault-resilient,
i.e., their tolerance to faults is workload-aware. In addition, a
novel population seeding mechanism to reduce the design space
is introduced and experimentally validated.

In summary, it is shown that SYFR can be considered a competitive
synthesis methodology for constructing fault-resilient circuits.
