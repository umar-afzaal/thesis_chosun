%----------------------------------------------------------------------------------------
%	Appendix Settings
%----------------------------------------------------------------------------------------
\setcounter{figure}{0}
\setcounter{table}{0}
\setcounter{algorithm}{0}
\setcounter{equation}{0}

\renewcommand{\thefigure}{\thesection.\arabic{figure}}
\renewcommand{\thetable}{\thesection.\arabic{table}}
\renewcommand{\thealgorithm}{\thesection.\arabic{algorithm}}
\renewcommand{\theequation}{\thesection.\arabic{equation}}

\refstepcounter{section}
%----------------------------------------------------------------------------------------

%----------------------------------------------------------------------------------------
%	Appendix Title
%----------------------------------------------------------------------------------------
\begin{center}
	\section*{APPENDIX \thesection : Synthesis and Characterization Tools}	
\end{center}
\addcontentsline{toc}{section}{APPENDIX \thesection: Synthesis and Characterization Tools}


\subsection{CVD System}

The CVD system mainly consists of quartz tube furnace, gas flow meters, gas pressure controller and rotary pump.
CVD furnace has following specifications.
\begin{enumerate}
\item Automatic nickel-chromium thermocouple which can control temperature up to 1100℃.
\item Quartz tube in the furnace has 80 mm diameter and 1500 mm length. Temperature is controlled by three thermocouples.%write more
\item Quartz jig where samples are placed.
\end{enumerate}

\subsubsection{Pneumatic Control and Vacuum System}

As a device for preparing composite materials by CVD method, its gas path system, flow meter and vacuum degree control system has a significant impact on the quality of the prepared graphene material. Among them, all digital flowmeters are calibrated with nitrogen (N2), the error rate is full the range is ± 1.5\% to ensure the accuracy of each reaction gas flow rate during the entire experiment. The overall structure of the gas circuit is connected by welded technology to ensure air tightness. A total of 3 gases including argon (Ar, 99.999\%), Hydrogen (H2, 99.999\%) and methane (CH4, 99.995\%) are designed to control the experiment conditions. The vacuum system consists of a first-stage diaphragm pump and a second-stage turbine and the ultimate vacuum can reach 1×10-8 mbar.

\subsection{Other Experimental Instruments}

Table \ref{tab:Other Experimental Instruments} shows the other experimental instruments and their usage
\begin{table}[!htb]
\centering
\caption{Instruments and their usage.}
\resizebox{\linewidth}{!}
{
\begin{tabular}{ccc}
\hline
Instrument          & Model                                 & Usage                      \\ \hline
Ultrasonic cleaner  & BRANSON  2210R-DTH                    & Wasshing and cleaning         \\
Drying oven         & HANBAEK SCIENTIFIC CO.,LTD  (HB-501M) & Drying \\
Vacuum glove box    & JISICO  J-924 AHO                     & Protection against moisture and oxidation \\
Polishing machine   & ALLIED TWINPREP 3 7M                  & Surface polishing to reveal microsrture \\
Analytical Balances & SHINKO DENSHICO.,LTD  (AF-R220E-D)    & Weighit the samples \\ \hline
\end{tabular}
}
\label{tab:Other Experimental Instruments}
\end{table}
\subsection{Characterization Methods}
Different instruments were employed in this study to characterize the materials fabricated and trace the reactions during the synthesis processes and are given in Table \ref{tab:characterization}. This section mainly introduces the graphene/hBN material structure characterization technology used in this work, and the material structure and the specific method characteristics of various physical properties are described in detail. These tools helped to optimize the synthesis conditions of the composites. 

\begin{table}[!htb]
\centering
\caption{Instruments and their usage.}
{
\begin{tabular}{cc}
\hline
Instrument & Model              \\ \hline
OM         & NIKON              \\
XRD        & PANalytical CubiX3 \\
SEM        & HITACHI S-4800     \\
TEM        & TECNAI G20FEI      \\ \hline
\end{tabular}
}
\label{tab:characterization}
\end{table}

